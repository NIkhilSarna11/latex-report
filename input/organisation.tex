\section{Introduction}
Automated Building Drawing is a project for creating two-dimensional drawings (front-view, top-view, side-view etc.) from a three-dimensional model.

The main purpose or objective of the project is to make it usable even by the layman. The main target users are the Civil Engineers who want their plans to be printed on the sheets. As of now, they have to create the drawings separately with different views in any CAD software and the 3D model separately. So to automate converting a particular three-dimensional model to the print-ready drawings (with different views), this project will be beneficial. The interface should be easy to use and pretty intuitive. Because the interface is a thing that makes user experience better and to make the user use it.

The Drawing module allows you to put your 3D work on paper. That is, to put views of your models in a 2D window and to insert that window in a drawing, for example a sheet with a border, a title and your logo and finally print that sheet. The drawing may consist of different views like top, front, side and orthographic views.

\section{Problem Formulation}

The main target users are the Civil Engineers who want their plans to be printed on the sheets. As of now, they have to create the drawings separately with different views in any CAD software and the 3D model separately. So to automate converting a particular three-dimensional model to the print-ready drawings (with different views), this project will be beneficial. So to decrease the efforts, time and cost, it would be really beneficial.

What we can do is to examine thorougly the existing systems if they work good or not. Another thing to do can be to fetch the values from a database (probably Object-orieneted). This will automate the whole process. As there are other tool to put the values of a 3D model into the database. And we can fetch those attributes and values from the database and create the drawings directly. 

First the aim is to make the life easier of a drafter. As if we need to view a model from a different angle and print it's drawing, then it would be a tedious task to do. Hence it can be considered as a further scope of improvements of existing systems.

\section{Objectives}
\begin{itemize}
\item To put views of your models in a 2D window and to insert that window in a drawing,.
\item Automatically creates orthographic views of an object.
\end{itemize}
\section{Feasibility Analysis}
Feasibility analysis aims to uncover the strengths and weaknesses of 
a project. In its simplest term, the two criteria to judge feasibility 
are cost required and value to be attained. As such, a well-designed 
feasibility analysis should provide a historical background of the 
project, description of the project or service, details of the 
operations and management and legal requirements. Generally, feasibility 
analysis precedes technical development and project implementation. 
There is some feasibility factors by which we can determine that 
project is feasible or not:
\begin{itemize}
\item {\bf{Technical feasibility}}: Technological feasibility is carried 
out to determine whether the project has the capability, in terms of 
software, hardware, personnel to handle and fulfill the user 
requirements. The assessment is based on an outline design of system 
requirements in terms of Input, Processes, Output and Procedures. Automated Building Drawings system is technically feasible as it is built up in Open 
Source Environment and thus it can be run on any Open Source plateform.
\item {\bf{Economic feasibility}}: Economic analysis is the most 
frequently used method to determine the cost/benefit factor for 
evaluating the effectiveness of a new system. In this analysis we 
determine whether the benefit is gained according to the cost invested 
to develop the project or not. If benefits outweigh costs, only then 
the decision is made to design and implement the system. It is 
important to identify cost and benefit factors, which can be categorized 
as follows:
\begin{enumerate}
\item Development costs.
\item Operating costs.
\end{enumerate}
Automated Building Drawings is also Economically feasible with 0 Development 
and Operating Charges as it is developed using open source technologies and the software is operated on Open 
Source platform.
\item {\bf {Legal feasibility}}: In this type of feasibility study, we 
basically determine whether the project conflicts with legal 
requirements, e.g. a data processing system must comply with the local 
Data Protection Acts. But the software has been developed with properly Licensed technologies. 
Thus is the legal process.
\item {\bf{Operational feasibility}}: Operational feasibility is a measure 
of how well a project solves the problems, and takes advantage of the 
opportunities identified during scope definition and how it satisfies 
the requirements identified in the requirements analysis phase of system 
development. All the operations performed in the system are very quick 
and satisfy all the requirements.
\item {\bf{Behaviour Feasibility}}: In this feasibility, we check about the 
behavior of the proposed system software i.e. whether the proposed 
project is user friendly or not, whether users can use the project 
without any training because of the user friendliness or not. Automated building drawings is very user friendly as its users interact with it 
through web.
\end{itemize}

\section{Methodology/Planning of work}
\begin{itemize}
\item Studying the current existing system and its problems.
\item Proposing solutions for various problems in the existing system.
\item Implementing the solutions and keeping in mind the benefits of the Automated building drawings system.
\end{itemize}

\section{Facilities required for proposed work}
\subsection{Hardware Requirements}
\begin{itemize}
\item Operating System: ubuntu 12.04 or windows 7
\item Processor Speed: 512KHz or more
\item RAM: Minimum 256MB
\end{itemize}
\subsection{Software Requirements}
\begin{itemize}
\item Software: Xampp or lampp(in case of ubuntu)
\item Programming Language: C++, Python, Qt
\item Database: MySQL or some Object-oriented database
\end{itemize}










