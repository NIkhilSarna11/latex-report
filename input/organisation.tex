\section{Introduction}
Automated Building Drawing is .\\

\noindent The traditional application of an LMS is in educational institutions. Learning management sys-
tems have been used for several years to deliver courseware in schools and popularize e-learning. In
the last few decades, Institutes have been using learning management systems to deliver training
to internal employees and students. The LMS has become a powerful tool for staffing and training,
extension schools, and any corporation looking to get a better grasp on the continuing education of
its workforce. Its impact has been felt mostly outside of traditional education institutions, though
the same technological and market forces are dramatically changing todays classroom as well.\\

\noindent Elearning is intended to provide methods or solutions to help our teachers as well as students
to improve internal processes, save time and increase efficiency. The use of computer systems to execute a variety of operations, such as adding downloadable
material, online quiz and tests, uploading and giving grades for assignments through internet
refers to what we call Elearning. The flexible nature of elearning means that we are likely to
encounter it in everyday life. Some people seek it out in for additional learning opportunities, and
for career advancement. While others may accidentally stumble upon it when watching a short
training on their smartphone about their latest application. The old adage still rings true, and
e-Learning brings with it new dimensions in education.\\

\noindent Elearning System is a user friendly system which can be used to access all the details about class by logging in. One can easily understand the website and use it for their convenience.
After login (using their roll number or name), several options for the students and teachers are available. As internet is used by nearly all the students and teachers so it is
very easy to use the system by the students as well as the teachers. It would be very helpful to the all the important people like
teachers and students who waste a lot of time and money my giving assignments and tests using pen and paper. Thus, using this system will add great relief in their busy life.
Along with the users, it is also very helpful for the institute administrator in the sense that they do
not have to carry all the student records everywhere, because everything will be available online, so they can access any information anytime.
\section{Problem Formulation}
Keeping track of daily work and performance of students and by the teachers is not an easy task in the current system, as they are not
connected to the database in any way. The traditional way of giving students assignments and other class related work is not very successfull in today's era as students mostly copy each other's assignments and other class work. Also the resources are being wasted
at each institute on manually updating data of students and teachers at various levels. There is no information about the
previous grades scores or assignments submitted by the students. Also, there is no proper track record recently pass out students. The teachers have to do a lot of pen and paper work on keeping record of students that who have have completed the class work assigned among all the students and who have not completed the work. Also the students have to wait for a significant amount of time after the submission of quizes or assignments to know their grade or marks.\\

\noindent The pen and paper work can be reduced to a great extent by doing work online and saving paper thus
making it environment friendly. Thus, making it automated process. Lot of time was wasted by the teachers in giving the assignments and later in collecting them from every student. The
previous process was quite complicated and long process. The students which have not submitted the work given to them can now be easily formulated by using this system.
Also it will be easy and less time consuming process to give students grades after any quiz they have completed. Now with system, we can easily know the status of students who have completed the given work and who have not.

\section{Objectives}
\begin{itemize}
\item To put views of your models in a 2D window and to insert that window in a drawing,.
\item Automatically creates orthographic views of an object.
\end{itemize}
\section{Feasibility Analysis}
Feasibility analysis aims to uncover the strengths and weaknesses of 
a project. In its simplest term, the two criteria to judge feasibility 
are cost required and value to be attained. As such, a well-designed 
feasibility analysis should provide a historical background of the 
project, description of the project or service, details of the 
operations and management and legal requirements. Generally, feasibility 
analysis precedes technical development and project implementation. 
There is some feasibility factors by which we can determine that 
project is feasible or not:
\begin{itemize}
\item {\bf{Technical feasibility}}: Technological feasibility is carried 
out to determine whether the project has the capability, in terms of 
software, hardware, personnel to handle and fulfill the user 
requirements. The assessment is based on an outline design of system 
requirements in terms of Input, Processes, Output and Procedures. Automated Building Drawings system is technically feasible as it is built up in Open 
Source Environment and thus it can be run on any Open Source plateform.
\item {\bf{Economic feasibility}}: Economic analysis is the most 
frequently used method to determine the cost/benefit factor for 
evaluating the effectiveness of a new system. In this analysis we 
determine whether the benefit is gained according to the cost invested 
to develop the project or not. If benefits outweigh costs, only then 
the decision is made to design and implement the system. It is 
important to identify cost and benefit factors, which can be categorized 
as follows:
\begin{enumerate}
\item Development costs.
\item Operating costs.
\end{enumerate}
Automated Building Drawings is also Economically feasible with 0 Development 
and Operating Charges as it is developed using open source technologies and the software is operated on Open 
Source platform.
\item {\bf {Legal feasibility}}: In this type of feasibility study, we 
basically determine whether the project conflicts with legal 
requirements, e.g. a data processing system must comply with the local 
Data Protection Acts. But the software has been developed with properly Licensed technologies. 
Thus is the legal process.
\item {\bf{Operational feasibility}}: Operational feasibility is a measure 
of how well a project solves the problems, and takes advantage of the 
opportunities identified during scope definition and how it satisfies 
the requirements identified in the requirements analysis phase of system 
development. All the operations performed in the system are very quick 
and satisfy all the requirements.
\item {\bf{Behaviour Feasibility}}: In this feasibility, we check about the 
behavior of the proposed system software i.e. whether the proposed 
project is user friendly or not, whether users can use the project 
without any training because of the user friendliness or not. Automated building drawings is very user friendly as its users interact with it 
through web.
\end{itemize}

\section{Methodology/Planning of work}
\begin{itemize}
\item Studying the current existing system and its problems.
\item Proposing solutions for various problems in the existing system.
\item Implementing the solutions and keeping in mind the benefits of the Automated building drawings system.
\end{itemize}

\section{Facilities required for proposed work}
\subsection{Hardware Requirements}
\begin{itemize}
\item Operating System: ubuntu 12.04 or windows 7
\item Processor Speed: 512KHz or more
\item RAM: Minimum 256MB
\end{itemize}
\subsection{Software Requirements}
\begin{itemize}
\item Software: Xampp or lampp(in case of ubuntu)
\item Programming Language: C++, Python, Qt
\item Database: MySql
\end{itemize}










